\documentclass[aspectratio=169]{beamer}
\newtheorem{proposition}{Proposition}
% \usepackage{appendixnumberbeamer}

%%% FIX ME:


\newcommand{\backupbegin}{                     %%
  \newcounter{finalframe}                      %%
  \setcounter{finalframe}{\value{framenumber}} %%
}                                              %%
%% 
\newcommand{\backupend}{                        %%
  \setcounter{framenumber}{\value{finalframe}} %%
}                                            %%


\setbeamertemplate{navigation symbols}{}
\newcommand*\oldmacro{}%
\let\oldmacro\insertshorttitle%
\renewcommand*\insertshorttitle{%
  \oldmacro\hfill%
  \insertframenumber\,/\,\inserttotalframenumber}
\usepackage{graphicx,hyperref}
\usepackage{epstopdf}
\usepackage{booktabs,calc}
\usepackage{ulem}
\usepackage{color,colortbl}
\usepackage{tikz}
\usepackage[utf8]{inputenc}
\usepackage{adjustbox}
\renewcommand{\labelitemii}{$\cdot$}
\renewcommand{\labelitemiii}{$\diamond$}
\renewcommand{\labelitemiv}{$\ast$}

%% SET LOCAL PATHS
\newcommand{\HOME}{\string~}

%% DON'T ADD NEW FILEPATH LINES -- CHANGE ~/include.tex INSTEAD
\input{\HOME/include.tex}

% define empty macro to control post regression table output
\newcommand{\postregsize}{}

% modify caption a little
\setbeamertemplate{caption}[numbered]
\setbeamerfont{caption}{size=\tiny}

\title[Sex ratios in India]{Preliminary work on sex ratios}
\author{DDL Internal} \institute[]{}{\normalsize
  \textcolor{gray}{}}

\date{September 2020}

\AtBeginSection[]{

  %%%%%%%%%%%%%%%%%%%%%%%%%%%%%%%%%%%%%%%%%%%%%%%%%%%%%%%%%%%%%%%%%%%%%%%%%%%%%%%%
  \begin{frame}
    \frametitle{Outline}
    \tableofcontents[currentsection,currentsubsection]
  \end{frame}
  %%%%%%%%%%%%%%%%%%%%%%%%%%%%%%%%%%%%%%%%%%%%%%%%%%%%%%%%%%%%%%%%%%%%%%%%%%%%%%%%
  
}
\addtobeamertemplate{navigation symbols}{}{ 
  \usebeamerfont{footline}%
  \usebeamercolor[fg]{footline}
  \hspace{1em}
  \insertframenumber/\inserttotalframenumber
}
\newenvironment{changemargin}[2]{% 
  \begin{list}{}{% 
      \setlength{\topsep}{#1}% 
      \setlength{\leftmargin}{#2}% 
      \setlength{\rightmargin}{12pt}% 
      \setlength{\listparindent}{\parindent}% 
      \setlength{\itemindent}{\parindent}% 
      \setlength{\parsep}{\parskip}% 
    }% 
  \item[]}{\end{list}} 

\begin{document}

\frame{\titlepage}
\end{frame}

\begin{section}{Data}
 %%%%%%%%%%%%%%%%%%%%%%%%%%%%%%%%%%%%%%%%%%%%%%%%%%%%%%%%%%%%%%%%%%%%%%%%%%%%%%%%
\begin{frame}{Raw Data}
\begin{table}[]
\centering
\resizebox{\linewidth}{!}{%
\begin{tabular}{lll}
\hline
&&\\
\textbf{Dataset} & \textbf{Contents} & \textbf{Limitation} \\
&&\\\hline
&& \\
Population Census & 1. PC 2011 Town directory has sex ratio & 1. Only covers urban areas. \\
 & for each town in 2011, 2000, 1991. &  \\ 
 & 2. PC 2000 has sex ratio for each town in 2000, 1991, and '81. & 2. Data is not available by age groups. \\
&&\\
\hline
&& \\
SECC & Household survey data with age and sex details & 1. Only available for 2012.\\
 & for each member, among other things. & 2. Coverage not as extensive as PC\\ 
 &  & (confirm this from Sam/Paul).\\ \bottomrule
\end{tabular}%
}
\end{table}
\end{frame}
%%%%%%%%%%%%%%%%%%%%%%%%%%%%%%%%%%%%%%%%%%%%%%%%%%%%%%%%%%%%%%%%%%%%%%%%%%%%%%%%
\begin{frame}{Clean Data}
  \begin{itemize}
    \item \textbf{PC sex ratios}: District\-wise sex ratios (urban areas only)
      in 2011, 2000, 1991, 1981 have been consolidated in
      \textcolor{blue}{/scratch/adibmk/pc\_sexratios}.  Note that the sex ratios for
      2011, 2000 and 1991 were pulled from PC11 town directory, and
      that for 1981 was pulled from PC01 town directory.
     \item \textbf{SECC district sex ratios by age group}: Currently saved in scratch folders of Aditi and Becky as \textcolor{blue}{secc\_sexratios\_age\_district}.
     \item \textbf{SECC sub\-district sex ratios by age group}: Currently saved in scratch folders of Aditi and Becky as \textcolor{blue}{secc\_sexratios\_age\_subdistrict}.
     \item \textbf{SECC birth order}: A dataset with birth order of each
       “child” (defined as age \<\= 18) in SECC households and age
       difference between each “child” has been saved separately for
       urban and rural areas as \textcolor{blue}{/scratch/adibmk/urban\_birthorder}, and
       \textcolor{blue}{/scratch/adibmk/rural\_birthorder} respectively.
     \item All the do files to create the clean datasets are currently sitting in \textcolor{blue}{flfp/subproject/b}
     \end{itemize}
\end{frame}
%%%%%%%%%%%%%%%%%%%%%%%%%%%%%%%%%%%%%%%%%%%%%%%%%%%%%%%%%%%%%%%%%%%%%%%%%%%%%%%%
%%%%%%%%%%%%%%%%%%%%%%%%%%%%%%%%%%%%%%%%%%%%%%%%%%%%%%%%%%%%%%%%%%%%%%%%%%%%%%%%
\end{section}
\begin{section}{Descriptive statistics}
  \begin{frame}{Validate SECC and PC sex ratio district data}
    \small{We used PC \emph{2011} data only as the SECC data is from 2012. We
    have also only used SECC \emph{urban} data as sex ratios are only
    available in PC town directories.}
    \centering
    \resizebox{0.8\linewidth}{!}{
    \begin{figure}
          \includegraphics[scale=0.8]{\pngpath/secc_pc11_district}
    \end{figure}
    }
  \end{frame}
  \begin{frame}{Validate SECC and PC sex ratio subdistrict data}
    \centering
    \resizebox{0.8\linewidth}{!}{
    \begin{figure}
          \includegraphics[scale=0.8]{\pngpath/secc_pc11_subdistrict}
    \end{figure}
    }
  \end{frame}

\begin{frame}{District level sex ratios in India across the four Census periods}
  \centering
    \resizebox{0.8\linewidth}{!}{
    \begin{figure}
          \includegraphics[scale=0.8]{\pngpath/sexratio_1}
    \end{figure}
    }
  \end{frame}
\begin{frame}{District level sex ratios in India across the four Census periods}
  \centering
    \resizebox{0.8\linewidth}{!}{
    \begin{figure}
          \includegraphics[scale=0.8]{\pngpath/sexratio_2}
    \end{figure}
    }
\end{frame}

\begin{frame}{Comparison with the literature}
  \centering
  The maps in the previous frames suggest that as of 1981, 1991, and
  2001, sex ratios were substantially lower than the natural sex ratio
  in North-West India compared to the rest of the country. This is in
  line with the fingind by Gupta, Monica Das, et al. in “Son Preference,
  Sex Ratios and ‘Missing Girls’ in Asia.” (Routledge Handbook of Asian
  Demography, 2018), which reports that while sex ratios have
  historically been more male-heavy in N India, sex selection is
  becoming a bigger problem is NW India, especially wealthier areas in
  Punjab and Haryana. See the figure on the next slide.
\end{frame}

\begin{frame}{Comparison with the literature}
  \centering
    \resizebox{0.6\linewidth}{!}{
    \begin{figure}
          \includegraphics[scale=0.6]{\pngpath/figure4}
    \end{figure}
    }
\end{frame}
\begin{frame}{Sex ratios are becoming male heavy over time in Indian
    cities, possibly reflecting urbanization and masculine rural to
    urban migration}
\centering
    \resizebox{0.8\linewidth}{!}{
    \begin{figure}
          \includegraphics[scale=0.8]{\pngpath/sex_timeseries}
    \end{figure}
    }
\end{frame}
\begin{frame}{Sex ratios by 5 year age bins: Sample graphs}
\centering
    \resizebox{0.8\linewidth}{!}{
    \begin{figure}
          \includegraphics[scale=0.8]{\pngpath/total_5_10}
    \end{figure}
    \begin{figure}
          \includegraphics[scale=0.8]{\pngpath/total_20_25}
    \end{figure}
    }
\end{frame}
\begin{frame}{Sex ratios by 5 year age bins: Sample graphs (Urban vs Rural)}
\centering
    \resizebox{0.8\linewidth}{!}{
    \begin{figure}
          \includegraphics[scale=0.8]{\pngpath/u_15_20}
    \end{figure}
    \begin{figure}
          \includegraphics[scale=0.8]{\pngpath/r_20_25}
    \end{figure}
    }
\end{frame}
\end{section}
\begin{section}{Analysis}
  \begin{frame}{Replicate Almond, Edlund (2008) finding of son-biased
      sex ratios in the United States using SECC for the Indian
      context. See reference graph from Almond, Edlund below: }
  \centering
    \resizebox{0.4\linewidth}{!}{
    \begin{figure}
          \includegraphics[scale=0.4]{\pngpath/reference}
    \end{figure}
    }
  \end{frame}
\begin{frame}{Replication in the Indian context using SECC
    \emph{urban} data: Evidence of son preference!}
  \centering
    \resizebox{0.6\linewidth}{!}{
    \begin{figure}
          \includegraphics[scale=0.6]{\pngpath/replication}
    \end{figure}
    }
  \end{frame}
\end{section}
\begin{section}{Next Steps}
\begin{frame}{Next Steps}
  \begin{itemize}
    \item Repeat the replication in the last slide for SECC rural
      areas to. The current graph only uses urban data.. because the
      rural dataset is huge. Ideas on how to do this efficiently with
      such a large dataset?
    \item Discuss what next with team.
    \item We have age-wise marriage data, enrollment data, and female
      employment at the village and town level.. how can they be
      combined with the SECC age wise sex ratios data to do something
      meaningful?
  \end{itemize}
\end{frame}
\end{section}
\end{document}
